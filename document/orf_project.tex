\documentclass{article}

\usepackage[version=3]{mhchem}
\usepackage[greek,english]{babel}
\usepackage{booktabs}
\usepackage{gensymb}
\usepackage{graphicx}
\usepackage[hidelinks]{hyperref}
\usepackage{fixltx2e}
\usepackage{titlesec}
\usepackage[normalem]{ulem}
\usepackage{parskip}
\usepackage{multirow}
\usepackage{fancyvrb}
\usepackage{color}
\usepackage{float}
\usepackage[super]{nth}
\usepackage{minted}

\titleformat*{\section}{\Large\bfseries}
\titleformat*{\subsection}{\bfseries}

\hypersetup{
  colorlinks   = true,
  urlcolor     = blue,
  linkcolor    = black,
  citecolor    = black
}

\definecolor{grey}{gray}{0.65}

\newfloat{listing}{thp}{lop}
\floatname{listing}{Listing}

\usemintedstyle{borland}

\setcounter{secnumdepth}{0}

\newcommand{\calpha}{C\textsubscript{\textgreek{a}}}
\newcommand{\ahelices}{\textgreek{a}-helices}
\newcommand{\bsheets}{\textgreek{b}-sheets}
\newcommand{\bbphi}{\ensuremath{\phi}}
\newcommand{\bbpsi}{\ensuremath{\psi}}
\newcommand{\bbomega}{\ensuremath{\omega}}
\newcommand{\module}[2]{\href{#2}{\texttt{#1}}}
\newcommand{\atomrec}{\texttt{ATOM} record}

\begin{document}

\title{Making Ramachandran Plots}
\author{Summer Programming Project}
\date{}
\maketitle{}

\section{Introduction}

The purpose of this project is to give you some practice using Python in a 
scientific context.  We designed the project for beginning programmers, but 
before starting you'll still have to teach yourself the basics of Python.  We 
recommend the book \emph{Learn Python the Hard Way} (LPTHW), which you can 
either \href{http://learnpythonthehardway.org/book/}{read for free online} or 
\href{https://paydiv.io/access/buy/2/}{purchase}.

The project is to write a program that generates Ramachandran plots from 
\href{http://www.rcsb.org/pdb/home/home.do}{Protein Data Bank} (PDB) files.  
Ramachandran plots show how backbone torsion angles are distributed in 
proteins.  Different features of proteins (namely \ahelices{} and \bsheets{}) 
occupy distinct regions in these plots.  For this reason, Ramachandran plots 
are commonly used both to validate experimental structures and to predict 
computational structures.

The project is divided into four parts:

\begin{enumerate}
 \item Setting up Python and reading the command line.
 \item Parsing groups of four backbone atoms from PDB files.
 \item Calculating torsions angles from those groups of atoms.
 \item Plotting those torsion angles using matplotlib.
\end{enumerate}

Try to get through at least one part every two weeks.  It takes time to get 
comfortable writing code, and you won't get there if you put off the whole 
project until just before bootcamp starts.  The project will be due on the 
first day of bootcamp.  We will wrap up the project with an activity where 
everyone explains their code to a partner (i.e. a code review).  This activity 
won't be as valuable for you if you don't finish the project on time.  If you 
are worried about finishing the project on time, or have any questions about 
the project or programming in general, please don't hesitate to email any one 
of us.

\subsection{What if you already know how to program?}

Even if you already know how to program, it's still important for you to work 
through this project so that you can participate in the code review we'll be 
holding on the first day of bootcamp (see above).  In fact, you should make an 
extra effort make your code as clean and understandable as possible so that it 
can be an example for the more novice programmers in your class.

If you know how to program, but not in Python, take this opportunity to learn 
Python.  The challenges in the \emph{Future Directions} section will be 
especially useful to you, because they will make you think a little bit more 
and really explore the language.

\end{document}
